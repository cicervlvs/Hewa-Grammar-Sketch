\documentclass{article}

\author{Alejandro García Matarredona}
\title{A Phonological Sketch of Hewa}

\usepackage[american]{babel}
%\usepackage[utf8]{inputenc}
\usepackage[T1]{fontenc}
\usepackage[english=american]{csquotes}

\usepackage[style=apa, backend=biber, natbib=true]{biblatex}
\addbibresource{C:/Users/aleja/Bibtex/library.bib}

\usepackage[american]{varioref}
\usepackage{tgtermes} %times font
\usepackage[T1]{tipa} %notació fonètica IPA
\usepackage{linguex} %exemples lingüístics

\usepackage{longtable}
\usepackage[table,xcdraw]{xcolor}
\usepackage{multirow} %taules

\usepackage{titlesec}
\usepackage{titletoc}
\setcounter{secnumdepth}{5}
\setcounter{tocdepth}{5}

\titleformat{\paragraph} [hang] {\normalfont\normalsize\bfseries} {\theparagraph} {1em} {}


\usepackage{titling}
\setlength{\droptitle}{-10em} 

\begin{document}

\maketitle

\section{Phonology}

In this section, the basic phonology of Hewa is outlined. In section \ref{sec:inv}, the consonant and vowel inventories are given. In section \ref{sec:spel}, the spelling system is presented. Section \ref{sec:contr} shows (near)- minimal pairs to give evidence for the contrasts among different phonemes. Section \ref{sec:syl} gives an overview of the syllable structure and phonotactics. Section \ref{sec:str} briefly discusses stress patterns. Finally, a list of the vocabulary compiled so far is presented.


\subsection{Phoneme inventory}\label{sec:inv}

\subsubsection{Consonants}\label{sec:cons}

Our consultant on Hewa shows 15 contrasting consonants, shown in table \ref{tab:cons} below. The affricate /\texttoptiebar{d\textipa{Z}}/ so far only appears in the lexical item \textit{meja} `table', which is a borrowing from Indonesian in turn borrowed from Portuguese.\footnote{See section \ref{sec:spel} for orthographic conventions.} 
 All consonants on the chart can appear in onset position (see section \ref{sec:syl} for the restrictions on coda position).\\

\begin{table}[h!]

\begin{tabular}{|c|c|c|c|c|c|}
\hline 
• & Bilabial & Dental/Alveolar & Postalveolar & Velar & Glottal \\ 
\hline 
Voiceless plosives & p & t & • & k & \textglotstop \\ 
\hline 
Voiced plosives & b & d & • & g & • \\ 
\hline 
Affricates & • & • & (\texttoptiebar{d\textipa{Z}}) & • & • \\ 
\hline 
Fricatives & \textbeta & s & • & • & h \\ 
\hline 
Nasals & m & n & • & \textipa{N} & • \\ 
\hline 
Laterals & • & l & • & • & • \\ 
\hline 
Rhotics & • & r & • & • & • \\ 
\hline 

\end{tabular} 
\caption{Contrasting consonants in Hewa}
\label{tab:cons}
\end{table} 

The dental plosives /t/ and /d/ appear as a retroflex [\textrtaild ] and [\textrtailt ] in certain positions; their distribution needs more research.

The rhotic /r/ seems to always appear as a trill. There is no contrast between a trilled rhotic /r/ and a tapped rhotic /\textfishhookr /.

The acoustic properties of the bilabial fricative /\textbeta / are unclear; further analysis could result in its reclassification as a labiodental approximant /\textscriptv /.

\paragraph{Consonant sequences and clusters}

Consonants rarely appear together within a word (see section \ref{sec:syl} as to why). The only two consonant combinations attested so far, be it  clusters or sequences, happening within the word are the clusters [pl], [bl], and [kl] in words such as \textit{plëka} `to cut', \textit{blon} `long', and \textit{kleren} `branches'.

More data is needed to describe how consonants act when in contact across word boundaries.


\subsubsection{Vowels}\label{sec:vow}

Our consultant shows six distinctive vowel sounds, with /\textschwa/ being restricted to the first syllable in between certain consonant combinations. All vowels become long in word initial position in open syllables, as well as in final position in monosyllabic words. \\

\begin{table}[h!]

\begin{center}
\begin{tabular}{|c|c|c|c|}
\hline 
• & Front & Central & Back \\ 
\hline 
High & i & • & u \\ 
\hline 
Mid & e & \textschwa & o \\ 
\hline 
Low & • & a & • \\ 
\hline 

\end{tabular} 
\caption{Contrasting vowels in Hewa}
\label{tab:vow}
\end{center}
\end{table}

% !!! there is one lexical item with a supposedly stressed ë: wë 'to go'

\paragraph{Vowel combinations}

Vowels often appear next to one another within a word, but rarely become a diphthong, instead often making up separate syllables, at least in careful speech. The only confirmed diphthong is in the word \textit{blau'} ([\textit{blaw\textglotstop}]) `to fear', which seems to be constrained by the final glottal stop. More data is needed to characterize the interaction of vowels across word boundaries.


\subsection{Spelling system}\label{sec:spel}

For the present sketch grammar, I have decided to design a simple orthography by which those sounds whose IPA representation does not correspond directly to a Latin alphabet grapheme are represented in the following way:

\begin{itemize}

\item <'> for the glottal stop /\textglotstop /
\item <w> for the bilabial fricative /\textbeta /
\item <ng> for the nasal velar /\textipa{N}/
\item <j> for the post-alveolar affricate /\texttoptiebar{d\textipa{Z}}/
\item <ë> for the mid central vowel /\textschwa /

\end{itemize}

All other sounds are represented with their Latin alphabet counterpart. I am using this orthographic system throughout the present sketch grammar unless specified with the use of square brackets or slashes for phonetic symbols.

\subsection{Phonemic contrasts}\label{sec:contr}

In this section I show some examples of phonemes the contrastiveness of which could in principle be argued.

\subsubsection{Initial and final glottal stops /\textglotstop/}

Word initial and word final glottal stops /\textglotstop / are not easily perceived by speakers of languages, such as most European ones, which do not have it as a constrastive phoneme. In Hewa, the contrast between the presence or absence of a glottal stop /\textglotstop / in word initial positions is clear when looking at vowel length: the vowel only becomes long in the absence of a consonant, not limited to, but including, the glottal stop /\textglotstop /.

\ex. \textit{a'u} ([\textipa{a:Pu}]) `I' vs \textit{'au} ([\textipa{Pau}]) `you (singular)'

Glottal stops in final position do not seem to affect the adjacent vowel in the same way as they do in initial position, but the speaker does point it out when repeating the elicited word back to him. A clear minimal pair is \ref{ex:ita}, which is a sequence of words that also happens to occur in natural speech.

\ex. \label{ex:ita} \textit{'ita} ([\textipa{Pita}]) `we (inclusive)' vs \textit{'ita'} ([\textipa{PitaP}]) `to see'
%Say exactly how this example makes a distinction

\subsubsection{Mid-front and central vowels /a/ /e/ /\textschwa/}

Below are some examples that show that /\textschwa / is a distinctive sound in Hewa. Example \ref{ex:blon} shows that [\textschwa] is not an epenthetical vowel, as it appears in between consonants which would otherwise form an acceptable cluster.

\ex. \label{ex:blon} \textit{bëli} ([\textipa{b@li}]) `to give' vs \textit{blon} ([blon]) `long'


% I might have my reasoning backwards: simplest explanation would be for the syllable to lose its stress first and then for /e/ to become /schwa/. But it only happens with /e/? What makes /e/ special in that regard?

Example \ref{ex:lap} shows a near-minimal pair between /\textschwa / and /a/; \ref{ex:rek} shows a near minimal pair between /\textschwa / and /e/.

\ex. \label{ex:lap} \textit{lëpo} ([\textipa{l@"po}]) `house' vs \textit{lapot}  ([\textipa{"lapot}])`to hit'

\ex. \label{ex:rek} \textit{rëkak} ([\textipa{r@"kak}]) `nether regions' vs \textit{rekat} ([\textipa{"rekat}]) `corner of the rice field'

\subsection{Syllable structure}\label{sec:syl}

%ALL BUT ONE MONOSYLLABIC WORD (tur) END WITH A LONG VOWEL

Almost all words in Hewa are monosyllabic or disyllabic . Tri- and quadrisyllabic words are attested but seem to be compounds; more research is needed in that regard. 

Below is a list of syllable patterns observed in the data gathered so far.

\begin{enumerate}

\item V (final syllable only) - \textit{wae}  ([\textipa{Ba.e}])`face'
%? (POSSIBLY GLOTTAL STOP THAT I'M MISSING)
\item CV - \textit{dula} ([du.la]) `belly'
\item VV - \textit{ala} ([a\textlengthmark .la]) `head'
\item CVV (monosyllabic words only) - \textit{to} [to\textlengthmark] `to laugh'
\item CVC (final syllable only) - \textit{lurin} ([lu.rin]) `bone' 
\item CCV - \textit{plupi} ([plu.pi]) `to blow'
\item CCVC (monosyllabic words only) - \textit{blon} ([blon]) `long'

% Change VV for V\textipa{:}. Not the same

\end{enumerate}

All Hewa consonants appear in onset position. The coda position is more restricted. Some of the restrictions observed are listed below.
% Make a table like in Phonology sketch.ppt with X for consonants attested in coda (also initial-medial above) and - for not attested.

\begin{itemize}

\item Plosives are restricted to /t/, /k/, and /\textglotstop /, which means that the voicing distinction is lost. The bilabial plosive /p/ only so far only appears in the data in the word \textit{lap} `to wipe', which is an Indonesian loan word, in turn borrowed by Indonesian from Dutch. It can then be argued that /p/ is in general not an acceptable coda in Hewa.

\item Fricatives do not appear in coda position, except for in loan words.

\item Nasals are restricted to /n/ and /\textipa{N}/, which means that the bilabial nasal /m/ is also not acceptable.

\item Liquids are restricted to /r/, which means that the contrast between /r/ and /l/ is lost and that /l/ is not acceptable.

\end{itemize}

The coda position can only be filled on the last syllable; examples of medial codas only appear in loan words (\textit{kursi} `chair'). Stressed syllables seem to require a weight of at least 2; in syllables without an onset or a coda, that is, in syllables where the only element is a vowel, said vowel is lengthened (e.g. \textit{ahu} [\textipa{a:hu}] `dog' vs \textit{gahu'} [\textipa{gahuP}] `hot'). The vowel is also lengthened in monosyllabic words with no coda, such as in  \textit{bo} ([\textipa{bo:}]) `to spit'. With these restrictions in mind, the syllable structure of a prototypical Hewa word can be summarized thus:

\begin{itemize}

\item A monosyllabic word may present the structure CVV (most commonly) or CVC (more rarely).

\item A disyllabic word may have as its first syllable CV, CCV, or VV, and as its second syllable V, CV, or CVC. The structure of the first syllable does not seem to condition that of the second one; however, words with the structure VV.V are not attested, and would be typologically strange.

\end{itemize}


\subsection{Stress}\label{sec:str}

%seems to be marked by a higher tone on the stressed syllable

Stress in Hewa appears to be predictable and therefore not contrasting. Words in Hewa are generally stress initial, except for when the first vowel is a /\textschwa/. It is unclear whether it is the /\textschwa/ that shifts the stress away from the syllable or whether /\textschwa/ is simply a result of a diachronic process of vocalic reduction stemming from a loss of stress on the syllable. The most likely explanation seems to be the latter, but this type of inquiry is beyond the scope of this sketch grammar.

\subsection*{Word list}\label{sec:wor}

Below is a list of the 187 lexical items compiled so far at this point in the research.
 
\begin{longtable}{|l|l|l|}
\hline
\textbf{Lexical item}   & \textbf{Gloss}      & {\color[HTML]{009901} \textbf{Category}}           \\ \hline
\endfirsthead
\textbf{Lexical item}   & \textbf{Gloss}      & {\color[HTML]{009901} \textbf{Category}}           \\ \hline
\endhead
apa                     & what                & {\color[HTML]{009901} Pronoun}                     \\ \hline
'ëra                    & stand               & {\color[HTML]{009901} Verb}                        \\ \hline
ahu                     & dog                 & {\color[HTML]{009901} Noun}                        \\ \hline
ala                     & head                & {\color[HTML]{009901} Noun}                        \\ \hline
anak                    & small               & {\color[HTML]{009901} Adjective}                   \\ \hline
awu                     & dust                & {\color[HTML]{009901} Noun}                        \\ \hline
a'u                     & I                   & {\color[HTML]{009901} Pronoun}                     \\ \hline
baka                    & bite                & {\color[HTML]{009901} Verb}                        \\ \hline
bëli                    & give                & {\color[HTML]{009901} Verb}                        \\ \hline
bëli mate               & kill                & {\color[HTML]{009901} Verb}                        \\ \hline
bërat                   & heavy               & {\color[HTML]{009901} Adjective}                   \\ \hline
bi'an                   & people              & {\color[HTML]{009901} Noun}                        \\ \hline
blara                   & sick                & {\color[HTML]{009901} Adjective}                   \\ \hline
blatan                  & cold                & {\color[HTML]{009901} Adjective}                   \\ \hline
blau'                   & fear                & {\color[HTML]{009901} Verb}                        \\ \hline
blon                    & long                & {\color[HTML]{009901} Adjective}                   \\ \hline
blosok                  & rub                 & {\color[HTML]{009901} Verb}                        \\ \hline
bo                      & spit                & {\color[HTML]{009901} Verb}                        \\ \hline
bokak                   & liver               & {\color[HTML]{009901} Noun}                        \\ \hline
bokat                   & mushroom            & {\color[HTML]{009901} Noun}                        \\ \hline
dedi' anak              & child               & {\color[HTML]{009901} Noun}                        \\ \hline
dedi' nurak             & newborn             & {\color[HTML]{009901} Noun}                        \\ \hline
dëhan                   & to tell on somebody & {\color[HTML]{009901} Verb}                        \\ \hline
dëka'                   & to flinch to sit    & {\color[HTML]{009901} Verb}                        \\ \hline
dëmen                   & correct             & {\color[HTML]{009901} Adjective}                   \\ \hline
dudun bëleng            & forest              & {\color[HTML]{009901} Noun}                        \\ \hline
dula                    & belly               & {\color[HTML]{009901} Noun}                        \\ \hline
du'a                    & wife                & {\color[HTML]{009901} Noun}                        \\ \hline
du'ur                   & to dry food         & {\color[HTML]{009901} Adjective}                   \\ \hline
e'on                    & not                 & {\color[HTML]{009901} Adverb}                      \\ \hline
gahu'                   & hot                 & {\color[HTML]{009901} Adjective}                   \\ \hline
ganupan                 & how                 & {\color[HTML]{009901} Pronoun}                     \\ \hline
gata'                   & to recite           & {\color[HTML]{009901} Verb}                        \\ \hline
gea                     & to eat (1PL)        & {\color[HTML]{009901} Verb}                        \\ \hline
                        & big                 & {\color[HTML]{009901} }                            \\ \cline{2-2}
\multirow{-2}{*}{gëte}  & wide                & \multirow{-2}{*}{{\color[HTML]{009901} Adjective}} \\ \hline
guman                   & night               & {\color[HTML]{009901} Noun}                        \\ \hline
guruk                   & still               & {\color[HTML]{009901} Adverb}                      \\ \hline
hai 'ia                 & who                 & {\color[HTML]{009901} Pronoun}                     \\ \hline
halo                    & hello               & {\color[HTML]{009901} Interjection}                \\ \hline
                        & many                & {\color[HTML]{009901} }                            \\ \cline{2-2}
\multirow{-2}{*}{harua} & a few               & \multirow{-2}{*}{{\color[HTML]{009901} Pronoun}}   \\ \hline
hika                    & split               & {\color[HTML]{009901} Verb}                        \\ \hline
hini                    & salt                & {\color[HTML]{009901} Noun}                        \\ \hline
hoka                    & to dig (with hoe)   & {\color[HTML]{009901} Verb}                        \\ \hline
horo                    & fly                 & {\color[HTML]{009901} Verb}                        \\ \hline
ina                     & mother              & {\color[HTML]{009901} Noun}                        \\ \hline
ina ama                 & parent              & {\color[HTML]{009901} Noun}                        \\ \hline
iru                     & nose                & {\color[HTML]{009901} Noun}                        \\ \hline
i'an                    & fish                & {\color[HTML]{009901} Noun}                        \\ \hline
kantar                  & sing                & {\color[HTML]{009901} Verb}                        \\ \hline
kapik                   & wing                & {\color[HTML]{009901} Noun}                        \\ \hline
kekor                   & feather             & {\color[HTML]{009901} Noun}                        \\ \hline
kënahoron               & bird                & {\color[HTML]{009901} Noun}                        \\ \hline
kera                    & sister's husband    & {\color[HTML]{009901} Noun}                        \\ \hline
kikir                   & fingernail          & {\color[HTML]{009901} Noun}                        \\ \hline
kleren                  & branches            & {\color[HTML]{009901} Noun}                        \\ \hline
krtas                   & paper               & {\color[HTML]{009901} Noun}                        \\ \hline
kursi                   & chair               & {\color[HTML]{009901} Noun}                        \\ \hline
lap                     & wipe                & {\color[HTML]{009901} Verb}                        \\ \hline
lapot                   & hit                 & {\color[HTML]{009901} Verb}                        \\ \hline
lau                     & that                & {\color[HTML]{009901} Demonstrative}               \\ \hline
lau 'ia                 & there               & {\color[HTML]{009901} Adverb}                      \\ \hline
lënung                  & pillar              & {\color[HTML]{009901} Noun}                        \\ \hline
lëpo                    & house               & {\color[HTML]{009901} Noun}                        \\ \hline
lëro                    & sun                 & {\color[HTML]{009901} Noun}                        \\ \hline
leten                   & stick               & {\color[HTML]{009901} Noun}                        \\ \hline
le'u wa'i ha            & all                 & {\color[HTML]{009901} Pronoun}                     \\ \hline
                        & hand                & {\color[HTML]{009901} }                            \\ \cline{2-2}
\multirow{-2}{*}{lima}  & arm                 & \multirow{-2}{*}{{\color[HTML]{009901} Noun}}      \\ \hline
lodoŋ                   & fall                & {\color[HTML]{009901} Verb}                        \\ \hline
lo'e                    & hair                & {\color[HTML]{009901} Noun}                        \\ \hline
lurin                   & bone                & {\color[HTML]{009901} Noun}                        \\ \hline
ma                      & tongue              & {\color[HTML]{009901} Noun}                        \\ \hline
manu                    & chicken             & {\color[HTML]{009901} Noun}                        \\ \hline
mata                    & eye                 & {\color[HTML]{009901} Noun}                        \\ \hline
mate ba'a               & die (already)       & {\color[HTML]{009901} Verb}                        \\ \hline
me1                     & newborn             & {\color[HTML]{009901} Noun}                        \\ \hline
me2                     & child (descendent)  & {\color[HTML]{009901} Noun}                        \\ \hline
me a'un du'at           & daughter            & {\color[HTML]{009901} Noun}                        \\ \hline
me a'un la'it           & son                 & {\color[HTML]{009901} Noun}                        \\ \hline
medja                   & table               & {\color[HTML]{009901} Noun}                        \\ \hline
mei                     & blood               & {\color[HTML]{009901} Noun}                        \\ \hline
meon                    & cat                 & {\color[HTML]{009901} Noun}                        \\ \hline
mun                     & close relatives     & {\color[HTML]{009901} Noun}                        \\ \hline
muta                    & vomit               & {\color[HTML]{009901} Verb}                        \\ \hline
nani                    & swim                & {\color[HTML]{009901} Verb}                        \\ \hline
napun                   & river               & {\color[HTML]{009901} Noun}                        \\ \hline
natar                   & village             & {\color[HTML]{009901} Noun}                        \\ \hline
nen                     & beach               & {\color[HTML]{009901} Noun}                        \\ \hline
nene'                   & grandparent         & {\color[HTML]{009901} Noun}                        \\ \hline
nene' ama               & grandfather         & {\color[HTML]{009901} Noun}                        \\ \hline
nene' puda mo'a         & grandparents        & {\color[HTML]{009901} Noun}                        \\ \hline
nene' wina              & grandmother         & {\color[HTML]{009901} Noun}                        \\ \hline
nian tana               & earth               & {\color[HTML]{009901} Noun}                        \\ \hline
nimu bi'an du'at        & she                 & {\color[HTML]{009901} Pronoun}                     \\ \hline
nimu bi'an la'it        & he                  & {\color[HTML]{009901} Pronoun}                     \\ \hline
niu                     & teeth               & {\color[HTML]{009901} Noun}                        \\ \hline
pano                    & walk                & {\color[HTML]{009901} Verb}                        \\ \hline
panolalan               & walk                & {\color[HTML]{009901} Verb}                        \\ \hline
papan                   & part                & {\color[HTML]{009901} Noun}                        \\ \hline
përa                    & squeeze             & {\color[HTML]{009901} Verb}                        \\ \hline
petak                   & rice field division & {\color[HTML]{009901} Noun}                        \\ \hline
pikr                    & think               & {\color[HTML]{009901} Verb}                        \\ \hline
plëka                   & cut                 & {\color[HTML]{009901} Verb}                        \\ \hline
plëmet                  & suck                & {\color[HTML]{009901} Verb}                        \\ \hline
pligo                   & hold                & {\color[HTML]{009901} Verb}                        \\ \hline
plupi                   & blow                & {\color[HTML]{009901} Verb}                        \\ \hline
popo                    & wash                & {\color[HTML]{009901} Verb}                        \\ \hline
pu                      & brother in law      & {\color[HTML]{009901} Noun}                        \\ \hline
puhun                   & flower              & {\color[HTML]{009901} Noun}                        \\ \hline
punu wi'in              & fight               & {\color[HTML]{009901} Verb}                        \\ \hline
pu'ur                   & short               & {\color[HTML]{009901} Adjective}                   \\ \hline
raha                    & chest               & {\color[HTML]{009901} Noun}                        \\ \hline
rakan                   & hunt                & {\color[HTML]{009901} Verb}                        \\ \hline
ramut                   & root                & {\color[HTML]{009901} Noun}                        \\ \hline
ra'itan                 & know                & {\color[HTML]{009901} Verb}                        \\ \hline
rëkak                   & nether region       & {\color[HTML]{009901} Noun}                        \\ \hline
rekat                   & rice field corner   & {\color[HTML]{009901} Noun}                        \\ \hline
rëmapira                & when                & {\color[HTML]{009901} Pronoun}                     \\ \hline
rëna                    & see                 & {\color[HTML]{009901} Verb}                        \\ \hline
reta                    & to                  & {\color[HTML]{009901} }                            \\ \hline
                        & you (pl)            & {\color[HTML]{009901} }                            \\ \cline{2-2}
\multirow{-2}{*}{rimu}  & they                & \multirow{-2}{*}{{\color[HTML]{009901} Pronoun}}   \\ \hline
robak                   & stab                & {\color[HTML]{009901} Verb}                        \\ \hline
roun                    & leaf                & {\color[HTML]{009901} Noun}                        \\ \hline
rumang                  & dark                & {\color[HTML]{009901} Adjective}                   \\ \hline
se1                     & lake                & {\color[HTML]{009901} Noun}                        \\ \hline
se2                     & go away             & {\color[HTML]{009901} Interjection}                \\ \hline
sëga'                   & come                & {\color[HTML]{009901} Noun}                        \\ \hline
senter                  & flashlight          & {\color[HTML]{009901} Verb}                        \\ \hline
sogor                   & push                & {\color[HTML]{009901} Verb}                        \\ \hline
tahi'                   & sea                 & {\color[HTML]{009901} Noun}                        \\ \hline
tali                    & rope                & {\color[HTML]{009901} Noun}                        \\ \hline
tana                    & ground              & {\color[HTML]{009901} Noun}                        \\ \hline
taran                   & horn                & {\color[HTML]{009901} Noun}                        \\ \hline
ta'idula                & guts                & {\color[HTML]{009901} Noun}                        \\ \hline
tëbo                    & body                & {\color[HTML]{009901} Noun}                        \\ \hline
tëgu                    & throw               & {\color[HTML]{009901} Verb}                        \\ \hline
tëlon                   & egg                 & {\color[HTML]{009901} Noun}                        \\ \hline
tëra                    & hard                & {\color[HTML]{009901} Adjective}                   \\ \hline
tëri                    & sit                 & {\color[HTML]{009901} Verb}                        \\ \hline
te'er                   & to wait             & {\color[HTML]{009901} Verb}                        \\ \hline
tilu                    & ear                 & {\color[HTML]{009901} Noun}                        \\ \hline
tinu                    & drink               & {\color[HTML]{009901} Verb}                        \\ \hline
to                      & laugh               & {\color[HTML]{009901} Verb}                        \\ \hline
to'e                    & back                & {\color[HTML]{009901} Noun}                        \\ \hline
tur                     & knee                & {\color[HTML]{009901} Noun}                        \\ \hline
tu'e                    & lay down            & {\color[HTML]{009901} Verb}                        \\ \hline
tu'e gëpaŋ              & sleep               & {\color[HTML]{009901} Verb}                        \\ \hline
ue                      & older brother       & {\color[HTML]{009901} Noun}                        \\ \hline
uhu                     & breast              & {\color[HTML]{009901} Noun}                        \\ \hline
ular                    & snake               & {\color[HTML]{009901} Noun}                        \\ \hline
ulit                    & bark                & {\color[HTML]{009901} Noun}                        \\ \hline
ulun                    & speak               & {\color[HTML]{009901} Verb}                        \\ \hline
uma                     & rice field house    & {\color[HTML]{009901} Noun}                        \\ \hline
unen                    & seed                & {\color[HTML]{009901} Noun}                        \\ \hline
upa                     & where               & {\color[HTML]{009901} Pronoun}                     \\ \hline
uran                    & rain                & {\color[HTML]{009901} Noun}                        \\ \hline
utu                     & louse               & {\color[HTML]{009901} Noun}                        \\ \hline
wa                      & mouth               & {\color[HTML]{009901} Noun}                        \\ \hline
wae                     & face                & {\color[HTML]{009901} Noun}                        \\ \hline
wair                    & water               & {\color[HTML]{009901} Noun}                        \\ \hline
wangak                  & flood               & {\color[HTML]{009901} Noun}                        \\ \hline
wari                    & younger brother     & {\color[HTML]{009901} Noun}                        \\ \hline
watu                    & stone               & {\color[HTML]{009901} Noun}                        \\ \hline
wa'an                   & grass               & {\color[HTML]{009901} Noun}                        \\ \hline
wa'i                    & leg                 & {\color[HTML]{009901} Noun}                        \\ \hline
wina                    & mother (intimate)   & {\color[HTML]{009901} Noun}                        \\ \hline
wine                    & sister              & {\color[HTML]{009901} Noun}                        \\ \hline
wongak1                 & distracted          & {\color[HTML]{009901} Adjective}                   \\ \hline
wongak2                 & to look up          & {\color[HTML]{009901} Verb}                        \\ \hline
wuan                    & fruit               & {\color[HTML]{009901} Noun}                        \\ \hline
wulan                   & moon                & {\color[HTML]{009901} Noun}                        \\ \hline
wu'an                   & heart               & {\color[HTML]{009901} Noun}                        \\ \hline
'ai                     & tree                & {\color[HTML]{009901} Noun}                        \\ \hline
'ai watu                & forest              & {\color[HTML]{009901} Noun}                        \\ \hline
'ali                    & dig                 & {\color[HTML]{009901} Verb}                        \\ \hline
'ami                    & 1pl.excl            & {\color[HTML]{009901} Pronoun}                     \\ \hline
'aro                    & scratch             & {\color[HTML]{009901} Verb}                        \\ \hline
'au                     & you (sg.)           & {\color[HTML]{009901} Pronoun}                     \\ \hline
'ea                     & eat                 & {\color[HTML]{009901} Verb}                        \\ \hline
'ëda                    & sand                & {\color[HTML]{009901} Noun}                        \\ \hline
'ëkak                   & offering            & {\color[HTML]{009901} Noun}                        \\ \hline
'ëla                    & fall                & {\color[HTML]{009901} Verb}                        \\ \hline
'ëru                    & neck                & {\color[HTML]{009901} Noun}                        \\ \hline
'ëtan                   & meat                & {\color[HTML]{009901} Noun}                        \\ \hline
'ete                    & this                & {\color[HTML]{009901} Demonstrative}               \\ \hline
'ita                    & we (incl)           & {\color[HTML]{009901} Pronoun}                     \\ \hline
'ita'                   & see                 & {\color[HTML]{009901} Verb}                        \\ \hline
'olon                   & bird                & {\color[HTML]{009901} Noun}                        \\ \hline
\end{longtable}

\end{document}
