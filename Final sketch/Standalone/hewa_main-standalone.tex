\documentclass{article}

\usepackage[american]{babel}
\usepackage[T1]{fontenc}
\usepackage[english=american]{csquotes}

\usepackage[style=apa, backend=biber, natbib=true]{biblatex}
\addbibresource{C:/Users/aleja/Bibtex/library.bib}
\addbibresource{Bibliography/misc.bib}

\usepackage[subpreambles=true]{standalone}
\usepackage{import} %com input, però millor per a standalone

\usepackage{mathptmx} %times font
\usepackage[T1]{tipa} %notació fonètica IPA
\usepackage{expex} %exemples lingüístics
\usepackage{multirow} %taules

\usepackage[american]{varioref}
\usepackage{hyperref}
\usepackage{cleveref}
\usepackage{graphicx}
\usepackage{caption}
\captionsetup{font=small} % fa els peus de pàgina més petits
\setlength{\textfloatsep}{22pt plus 1.0pt minus 2.0pt}


\usepackage{titlesec}
\usepackage{titletoc}
\setcounter{secnumdepth}{5}
\setcounter{tocdepth}{5}

\titleformat{\paragraph} [hang] {\normalfont\normalsize\bfseries} {\theparagraph} {1em} {}

\title{A sketch grammar of Hewa}

\begin{document}

\tableofcontents

%\makeglossaries

\section{Introduction}
\import{Sections/Introduction}{Introduction-final}

\section{Phonology}
\import{Sections/Phonology}{Phonology-safe}

\section{Morpho-syntax}
\import{Sections/Morpho-syntax}{Morpho-Syntax-final}

\appendix
\section{Hewa text: description of activities to do in and around the Village of Hewa}

\section{Hewa wordlist}

\printbibliography

\end{document}
