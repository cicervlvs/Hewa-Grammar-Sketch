\documentclass[../hewa_main-subfiles.tex]{subfiles}

\begin{document}

\section{Phonology}

In this section, the basic phonology of Hewa is outlined. In section \ref{sec:inv}, the consonant and vowel inventories are given. In section \ref{sec:spel}, the spelling system is presented. Section \ref{sec:contr} shows (near)- minimal pairs to give evidence for the contrasts among different phonemes. Section \ref{sec:syl} gives an overview of the syllable structure and phonotactics. Finally, Section \ref{sec:str} briefly discusses stress patterns.

\subsection{Phoneme inventory}\label{sec:inv}

\subsubsection{Consonants}\label{sec:cons}

Our consultant on Hewa shows 15 contrasting consonants, shown in table \ref{tab:cons} below. The affricates /\texttoptiebar{t\textipa{S}}/ and /\texttoptiebar{d\textipa{Z}}/ so far only appear in lexical items borrowed from Indonesian, such as \textit{coba} `try' and \textit{meja} `table'.\footnote{See section \ref{sec:spel} for orthographic conventions.} 
 All consonants on the chart can appear in onset position (see section \ref{sec:syl} for the restrictions on coda position).

\begin{table}[h!]
\centering
\caption{Contrasting consonants in Hewa}
\label{tab:cons}
\begin{tabular}{@{}lccccc@{}}
\toprule
 & Bilabial & Dental/Alveolar & Postalveolar & Velar & Glottal \\ \midrule
Voiceless plosives & p & t &  & k & \textglotstop \\
Voiced plosives & b & d &  & g &  \\
Affricates &  &  & (\texttoptiebar{t\textipa{S}}) (\texttoptiebar{d\textipa{Z}}) &  &  \\
Fricatives & \textbeta & s &  &  & h \\
Nasals & m & n &  & \textipa{N} &  \\
Laterals &  & l &  &  &  \\
Rhotics &  & r &  &  &  \\ \bottomrule
\end{tabular}
\end{table}

The dental plosives /t/ and /d/ seem to appear as a retroflex [\textrtaild ] and [\textrtailt ] or as interdental [\textsubbridge{t}] and [\textsubbridge{d}] at certain times; their distribution is unclear.

The rhotic /r/ seems to always appear as a trill. There is no contrast between a trilled rhotic /r/ and a tapped rhotic /\textfishhookr /.

The acoustic properties of the bilabial fricative /\textbeta / are unclear; further analysis could result in its reclassification as a labiodental approximant /\textscriptv /.

\paragraph{Consonant sequences and clusters}

Consonants rarely appear together within a word (see section \ref{sec:syl} as to why). The only two consonant combinations attested so far, be it  clusters or sequences, happening within the word are the clusters [pl], [bl], and [kl] in words such as \textit{plëka} `to cut', \textit{blon} `long', and \textit{kleren} `branches'.

There does not seem to be any particular interaction between consonants when in contact across word boundaries.

\subsubsection{Vowels}\label{sec:vow}

Our consultant shows six distinctive vowel sounds, shown in table \ref{tab:vow}.

\begin{table}[h!]
\centering
\caption{Contrasting vowels in Hewa}
\label{tab:vow}
\begin{tabular}{@{}llll@{}}
\toprule
 & Front & Central & Back \\ \midrule
High & i &  & u \\
Mid & e & \textschwa * & o \\
Low &  & a &  \\ \bottomrule
\end{tabular}
\end{table}

The mid-central vowel /\textschwa/ is presented with an asterisk because it has a restricted distribution and an uncertain status, discussed briefly in section \ref{sec:contr}. All vowels become long in word initial position in open syllables, as well as in final position in monosyllabic words. \\

% !!! there is one lexical item with a supposedly stressed ë: wë 'to go'

\paragraph{Vowel combinations}

Vowels often appear next to one another within a word, but rarely become a diphthong, instead often making up separate syllables, at least in careful speech. The only confirmed diphthong is in the word \textit{blau'} ([\textit{blaw\textglotstop}]) `to fear', which seems to be constrained by the final glottal stop. There does not appear to be an interaction of vowels across word boundaries.

\subsection{Spelling system}\label{sec:spel}

For the present sketch grammar, I have decided to design a simple orthography by which those sounds whose IPA representation does not correspond directly to a Latin alphabet grapheme are represented in the following way:

\begin{itemize}

\item <'> for the glottal stop /\textglotstop /
\item <w> for the bilabial fricative /\textbeta /
\item <ng> for the nasal velar /\textipa{N}/
\item <c> for the voiceless post-alveolar affricate /\texttoptiebar{t\textipa{S}}/
\item <j> for the post-alveolar affricate /\texttoptiebar{d\textipa{Z}}/
\item <ë> for the mid central vowel /\textschwa /

\end{itemize}

All other sounds are represented with their Latin alphabet counterpart. I am using this orthographic system throughout the present sketch grammar unless specified with the use of square brackets or slashes for phonetic symbols.

\subsection{Phonemic contrasts}\label{sec:contr}

In this section I show some examples of phonemes the contrastiveness of which could in principle be argued.

\subsubsection{Initial and final glottal stops /\textglotstop/}

Word initial and word final glottal stops /\textglotstop / are not easily perceived by speakers of languages, such as most European ones, which do not have it as a constrastive phoneme. In Hewa, the contrast between the presence or absence of a glottal stop /\textglotstop / in word initial positions is clear when looking at vowel length: the vowel only becomes long in the absence of a consonant, not limited to, but including, the glottal stop /\textglotstop /.

\ex<glotstop> %% "main" example needs a tag
\textit{a'u} ([\textipa{a:Pu}]) `I' vs \textit{'au} ([\textipa{Pau}]) `you (singular)'
\xe


Glottal stops in final position do not seem to affect the adjacent vowel in the same way as they do in initial position, but the speaker does point it out when repeating the elicited word back to him. A clear minimal pair is (\getfullref{ita}), which is a sequence of words that also happens to occur in natural speech.

\ex<ita> \textit{'ita} ([\textipa{Pita}]) `we (inclusive)' vs \textit{'ita'} ([\textipa{PitaP}]) `to see'
\xe
%Say exactly how this example makes a distinction

In this example, the only difference between the first and the second word is the presence of a final glottal stop in final position.

\subsubsection{alveolar nasal /n/ and velar nasal /\textipa{N}/}

As explained in section \ref{sec:syl} below, the velar nasal \textipa{N} shows a limited distribution, as it does not appear in word initial position. It is the only Hewa phoneme not to do so. Regardless, it may be considered a full phoneme as it appears in all other positions, that is, in word-medial onsets and in codas. Furthermore, (\getfullref{nas}) shows a near-minimal pair with the alveolar nasal /n/.

\ex<nas> %% "main" example needs a tag
\textit{wangak} ([\textipa{"BaNak}]) `flood' vs \textit{anak} ([\textipa{"anak}]) `child'
\xe

\subsubsection{On the status of /\textschwa/}

The exact status and distribution of the mid-central vowel [\textschwa] remains unclear as of the final writing of this report. It generally only appears in initial unstressed syllables, and always with a consonantal onset. There is one lexical item in the data, \textit{wë} ([\textipa{B@:}]) `to go', which has a [\textschwa] in a stressed position. This word makes a minimal pair with  \textit{wa} ([\textipa{Ba:}]) `mouth', but perhaps remarkably there is no combination of /\textbeta / and /e/ in the data to contrast \textit{wë} with.

Example (\getfullref{lap}) shows a near-minimal pair between /\textschwa / and /a/; (\getfullref{rek}) shows a near minimal pair between /\textschwa / and /e/.

\ex<lap> \textit{lëpo} ([\textipa{l@"po}]) `house' vs \textit{lapot}  ([\textipa{"lapot}])`to hit'
\xe

\ex<rek> \textit{rëkak} ([\textipa{r@"kak}]) `nether regions' vs \textit{rekat} ([\textipa{"rekat}]) `corner of the rice field'
\xe

However, in these examples the stress shifts between initial in the case of the words with /a/ and /e/ and final in the case of the words with /\textschwa /. As explained above, [\textschwa] only ever appears in unstressed syllables, with the aforementioned exception of \textit{wë} `to go'. It is possible that the examples with [\textschwa] in an unstressed syllable are words which have shifted	 their stress, with possibly an underlying /e/ or /a/ which with the loss of stress has become [\textschwa]. This hypothesis would need more explaining, as there are numerous examples of /e/ and /a/ in unstressed syllables, where they seem to behave like the rest of the vowels.

Regardless of its particular phonological status, it is evident that [\textschwa] is never an epenthetic vowel, as shown in (\getfullref{blon}).

\ex<blon> \textit{bëli} ([\textipa{b@li}]) `to give' vs \textit{blon} ([blon]) `long'
\xe

Here, [\textschwa] appears in between consonants which would otherwise form an acceptable cluster, which means that it has not been added epenthetically to the words in which it appears.

% I might have my reasoning backwards: simplest explanation would be for the syllable to lose its stress first and then for /e/ to become /schwa/. But it only happens with /e/? What makes /e/ special in that regard?


\subsection{Syllable structure}\label{sec:syl}

%ALL BUT ONE MONOSYLLABIC WORD (tur) END WITH A LONG VOWEL

Almost all words in Hewa are monosyllabic or disyllabic. Tri- and quadrisyllabic words are attested only in compounds.

Below is a list of syllable patterns observed in the data gathered so far.

\begin{enumerate}

\item V (final syllable only) - \textit{wae}  ([\textipa{Ba.e}])`face'
\item CV - \textit{dula} ([du.la]) `belly'
\item V\textlengthmark\ - \textit{ala} ([a\textlengthmark .la]) `head'
\item CV\textlengthmark\ (monosyllabic words only) - \textit{to} [to\textlengthmark] `to laugh'
\item CVC (final syllable only) - \textit{lurin} ([lu.rin]) `bone' 
\item CCV - \textit{plupi} ([plu.pi]) `to blow'
\item CCVC (monosyllabic words only) - \textit{blon} ([blon]) `long'

\end{enumerate}


All Hewa consonants appear in onset position, and all but /\textipa{N}/ appear in word initial position. This is shown in table \ref{tab:cons-onset}.

\begin{table}[h!]
\caption{Hewa consonants as onset (x = attested, - = unattested, * = only in loan words)}
\label{tab:cons-onset}
\begin{tabular}{@{}llllllllllllllllll@{}}
\toprule
Onset & p & b & t & d & k & g & \textglotstop & (\texttoptiebar{t\textipa{S}}) & (\texttoptiebar{d\textipa{Z}}) & \textbeta & s & h & m & n & \textipa{N} & l & r \\ \midrule
Word-initial & x & x & x & x & x & x & x & x* & x* & x & x & x & x & x & \cellcolor[HTML]{EFEFEF}- & x & x \\
Word-medial & x & x & x & x & x & x & x & x* & x* & x & x & x & x & x & \cellcolor[HTML]{EFEFEF}x & x & x \\ \bottomrule
\end{tabular}
\end{table}

The coda position is more restricted when compared to the onset position. Table \ref{tab:cons-coda} shows the distribution of consonants in coda position.

\begin{table}[h!]
\caption{Hewa consonants as coda (x = attested, - = unattested, * = only in loan words)}
\label{tab:cons-coda}
\begin{tabular}{@{}
>{\columncolor[HTML]{FFFFFF}}l 
>{\columncolor[HTML]{FFFFFF}}l 
>{\columncolor[HTML]{FFFFFF}}l 
>{\columncolor[HTML]{FFFFFF}}l 
>{\columncolor[HTML]{FFFFFF}}l 
>{\columncolor[HTML]{FFFFFF}}l 
>{\columncolor[HTML]{FFFFFF}}l 
>{\columncolor[HTML]{FFFFFF}}l 
>{\columncolor[HTML]{FFFFFF}}l 
>{\columncolor[HTML]{FFFFFF}}l 
>{\columncolor[HTML]{FFFFFF}}l 
>{\columncolor[HTML]{FFFFFF}}l 
>{\columncolor[HTML]{FFFFFF}}l 
>{\columncolor[HTML]{FFFFFF}}l 
>{\columncolor[HTML]{FFFFFF}}l 
>{\columncolor[HTML]{FFFFFF}}l 
>{\columncolor[HTML]{FFFFFF}}l 
>{\columncolor[HTML]{FFFFFF}}l @{}}
\toprule
Coda & p & b & t & d & k & g & \textglotstop & (\texttoptiebar{t\textipa{S}}) & (\texttoptiebar{d\textipa{Z}}) & \textbeta & s & h & m & n & \textipa{N} & l & r \\ \midrule
 & x* & \cellcolor[HTML]{EFEFEF}- & x & \cellcolor[HTML]{EFEFEF}- & x & \cellcolor[HTML]{EFEFEF}- & x & \cellcolor[HTML]{EFEFEF}- & \cellcolor[HTML]{EFEFEF}- & \cellcolor[HTML]{EFEFEF}{\color[HTML]{000000} -} & x* & \cellcolor[HTML]{EFEFEF}- & x* & x & x & \cellcolor[HTML]{EFEFEF}- & x \\ \bottomrule
\end{tabular}
\end{table}

Given this distribution, some preliminary rules may be established. These tentative restrictions are listed below.

\begin{itemize}

\item Plosives are restricted to /t/, /k/, and /\textglotstop /, which means that the voicing distinction is absent. The bilabial plosive /p/ only appears in the data in the word \textit{lap} `to wipe', which is an Indonesian loan word, in turn borrowed by Indonesian from Dutch. It can then be argued that /p/ is in general not an acceptable coda in Hewa.

\item Fricatives (/\textbeta / , /s/, /h/)  do not appear in coda position, except in loan words.

\item Nasals are restricted to /n/ and /\textipa{N}/, which means that the bilabial nasal /m/ is also not acceptable.

\item Liquids are restricted to /r/, which means that the contrast between /r/ and /l/ is absent and that /l/ is not acceptable.

\end{itemize}

The coda position can only be filled in the last syllable; as all initial syllables in multisyllabic words are open. Instances of medial codas only appear in loan words (\textit{kursi} `chair'). Stressed syllables seem to require a weight of at least 2; in syllables without an onset or a coda, that is, in syllables where the only element is a vowel, said vowel is lengthened (e.g. \textit{ahu} [\textipa{a:hu}] `dog' vs \textit{gahu'} [\textipa{gahuP}] `hot'). The vowel is also lengthened in monosyllabic words with no coda, such as in  \textit{bo} ([\textipa{bo:}]) `to spit'. With these restrictions in mind, the syllable structure of a prototypical Hewa word can be summarized thus:

\begin{itemize}

\item A monosyllabic word may present the structure CV\textipa{:} (most commonly) or CVC (more rarely).

\item A disyllabic word may have as its first syllable CV, CCV, or V\textlengthmark , and as its second syllable V, CV, or CVC. The structure of the first syllable does not seem to condition that of the second one; however, words with the structure V\textlengthmark .V are not attested, and would be typologically strange.

\end{itemize}

\subsection{Stress}\label{sec:str}

%seems to be marked by a higher tone on the stressed syllable

Stress in Hewa appears to be predictable and therefore not contrasting. Words in Hewa are generally stress initial, except for when the first vowel is a /\textschwa/. It is unclear whether it is the /\textschwa/ that shifts the stress away from the syllable or whether /\textschwa/ is simply a result of a diachronic process of vocalic reduction stemming from a loss of stress on the syllable. The latter explanation seems the most likely (see \ref{sec:contr} for a brief discussion), but this type of inquiry is beyond the scope of this sketch grammar.

\end{document}
